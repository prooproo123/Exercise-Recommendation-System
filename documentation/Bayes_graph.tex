\documentclass{report}
\usepackage{amsmath}
\usepackage{hyperref}
\usepackage{titlesec}


\titleformat{\chapter}[display]
{\normalfont\bfseries}{}{0pt}{\Large}

\begin{document}
	\chapter{Uvod}
	
	Ideja je napraviti vlastiti model koji iz skupa podataka računa vjerojatnost savladavanja koncepata te se prema njima gradi graf znanja.
	
	Prvi pristup je račun uvjetnih vjerojatnosti $p(Y_j\mid X_i)$
	
	Najveći problem u oba pristupa je to što pokušavamo dobiti ovisnosti među konceptima,a ne pitanjima. Za razliku od pitanja za koncepte se ne može reći da ih se zna/ne zna jer se oni sastoje od više pitanja te se može samo gledati postotak riješenosti za svakog studenta / prosječan postotak riješenosti. Prosječan postotak rješenosti se može gledati kao pripadnost neizrazitom skupu ZNA odnosno 1- postotak riješenosti kao pripadnost skupu NE ZNA, to komplicira stvari kod Bayesovog zaključka. Potrebno je pronaći/proučiti postoji li ekvivalent Bayesovog zaključka kada se koriste neizraziti skupovi odnosno kontinuirane vrijednosti [0,1].
	
\end{document}
