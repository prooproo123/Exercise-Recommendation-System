\documentclass[times, utf8,projekt]{fer}
\usepackage{booktabs}
\usepackage{amsmath}
\usepackage{hyperref}
\usepackage{titlesec}
\begin{document}

% TODO: Navedite broj rada.
%\thesisnumber{000}

% TODO: Navedite naslov rada.
\title{Strojno učenje u adaptivnom učenju}

% TODO: Navedite vaše ime i prezime.
\author{Jelena Nemčić, Marin Ovčariček,Zvonimir Sučić,Ivana Žeger}

\maketitle
% Ispis stranice s napomenom o umetanju izvornika rada. Uklonite naredbu \izvornik ako želite izbaciti tu stranicu.
%\izvornik
% Dodavanje zahvale ili prazne stranice. Ako ne želite dodati zahvalu, naredbu ostavite radi prazne stranice.
%\zahvala{}

\tableofcontents
%Dodati općenitu ideju projekta
\chapter{Uvod}
Uvod rada. Nakon uvoda dolaze poglavlja u kojima se obrađuje tema.

\chapter{Razrada teme}
\chapter{Bayesian knowledge tracing (BKT)}
\section{Općenito o BKT}
Bayesian Knowledge Tracing koristi Hidden Markov Model i ima 4 osnovna parametra:
\begin{itemize}
	\item p(L0) - vjerojatnost da je korisnik a priori savladao gradivo
	\item p(G) - vjerojatnost da je korisnik pogodio točan odgovor bez da ima potrebno znanje	
	\item p(S) - vjerojatnost da je korisnik krivo odgovorio iako ima potrebno znanje	
	\item p(T) - vjerojatnost da je znanje prešlo iz NE ZNA u ZNA nakon prilike da se primjeni znanje
\end{itemize}
Kao izlaz dobivaju se vrijednosti:
\begin{itemize}
	\item p(L) - vjerojatnost ovladavanja vještinom (eng. probability of skill mastery)
	\item p(C) - vjerojatnost da će korisnik ispravno primijeniti vještinu u budućnosti (eng. probability of the student correctly applying the skill on a future practice)\newline
\end{itemize}

\begin{equation}
 p(L_t\mid obs=correct)=\frac{p(L_t)*(1-p(S))}{p(L_t)*(1-p(S)+(1-p(L_t)*p(G))}
\end{equation}\newline
\begin{equation}
 p(L_t\mid obs=wrong)=\frac{p(L_t)*p(S)}{p(L_t)*p(S)+(1-p(L_t)*(1-p(G))}
\end{equation}\newline
\begin{equation}
 p(L_{t+1})=p(L_t\mid obs=correct) + (1-p(L_t\mid obs=correct))*p(T)
\end{equation}\newline
\begin{equation}
 p(C_{t+1})=p(L_{t+1}) * (1-p(S)) + p(L_{t+1})*p(G)
\end{equation}

\section{Ideje}
Prvobitna ideja je bila da se p(L0) računa iz inicijalnih pitanja, vrijednosti p(G) i p(S) bi se prema preporuci iz rada(trebalo bi pronaći kojeg i baciti referencu) stavile na interval [0,0.3], [0,0.1] te bi se p(T) postavio prema preporuci eksperta što ne želimo jer je cilj ovog projekta da smanjimo zadatke eksperata na minimum.

To je ukazalo na potrebu pronalaska algoritama koji bi uz pomoć nekog skupa podataka aproksimirali parametre za BKT.
\section{Problemi i zadaci}
\begin{itemize}
	\item proučiti parametar p(T)
	\item proučiti kodove sa githuba kako bi se dobila ideja kako algoritam funkcionira
	\item napraviti malu implementaciju s malo pitanja i provjeriti radi li
	\item proučiti parameter fitting uz pomoć EM algoritma, stohastic gradient descenta ili neke druge metode
	\item kako napraviti input dataset, prikupiti podatke
	
	
\end{itemize}
\section{Dobivanje BKT parametara}
\subsection{EM (expectation-maximization) algoritam}
\begin{itemize}
	\item iterativni algoritam za pronalaženje (aproksimiranje) najveće izglednosti (eng. maximum likelihood) ili maksimalne a posteriori (MAP) procjene parametara u statističkim modelima
	\item model ovisi o nepoznatim latentnim varijablama
	\item EM iteracija sadrži 2 koraka:
	\begin{itemize}
		\item 	korak očekivanja (E), koji stvara funkciju za očekivanje log-izglednosti koja se procjenjuje pomoću trenutne procjene parametara, procjenjuju se vrijednosti latentnih varijabli
		\item 	korak maksimizacije (M), koji izračunava parametre distribucije koji maksimiziraju očekivanu log-izglednost pronađenu u E koraku, ti se parametri zatim koriste za procjenu latentnih varijabli u sljedećem E koraku
	\end{itemize}
	
	\item primjenjuje se kada želimo odrediti parametre distribucije (normalna, eksponencijalna, …)
	\item problem: za korištenje potrebo znati distribuciju podataka ili točne vrijednosti (eng. true values) traženih parametara
	
\end{itemize}
Kroz ovo istraživanje nije pronađena niti jedna implementacija EM algoritma za aproksimaciju BKT parametara niti je napravljena vlastiti implementacija zbog prevelikog praga znanja matematike.

\subsection{Grid search i Simulated Annealing}
Pronađen je kod napisan u Javi koji računa BKT parametre tehnikom simuliranog kaljenja \url{https://github.com/wlmiller/BKTSimulatedAnnealing}. U README na githubu se također spominjao kod koji je bio baza za to, on je koristio običan grid search kako bi izračunao parametre. Oba koda su prevedena u python i prilagođena našim skupovima podataka. Na kraju se ispostavilo da je "simulirano kaljenje" povoljnije te se grid search odbacio.
\section{Rezultati}
\begin{itemize}
	\item napravljen google forms kviz sa 20 pitanja iz biologije, ispitanici moraju odgovoriti na svih 20 pitanja kako bi podaci ušli u dataset
	\item napravljena python skripta koja pretvara podatke dobivene iz google formsa u oblik prikladan za treniranje BKT-a i pronalaženje parametara
	\item pronađen je kod u Javi koji tehnikom simuliranog kaljenja aproksimira parametre za BKT uz pomoć danog dataseta, kod je preveden u python skript
	\item napravljena python skripta za BKT koja određuje vjerojatnost da je ispitanik naučio/  savladao gradivo
	\item uz pomoć skripte za aproksimaciju BKT parametara, nađene su njihove vrijednosti za svaku vještinu iz ASSISTMENTS dataseta i pohranjenje u google sheets tablicu
	\item dobiveni parametri algoritmom simuliranog kaljenja uspoređeni su s onima dobivenima pomoću grid search metode -> vrijednosti parametara su skoro iste, vrlo male razlike
	\item napravljen google forms kviz sa po 6 pitanja iz 5 koncepata, izračunati su parametri za taj dataset
	\item BKT kod i kod za aproksimaciju BKT parametara su se dalje koristili u bilježnicama za izgradnju grafa probabilističkim metodama
	
\end{itemize}
\section{Poveznice}
\subsection{BKT}
\url{https://en.wikipedia.org/wiki/Bayesian_Knowledge_Tracing}\newline
\url{http://www.cs.cmu.edu/~./ggordon/yudelson-koedinger-gordon-individualized-bayesian-knowledge-tracing.pdf}\newline
\url{https://github.com/CAHLR/pyBKT/blob/master/README.md}\newline
\url{https://www.learnlab.org/uploads/mypslc/publications/bca2008v.pdf}\newline
\url{https://www.upenn.edu/learninganalytics/ryanbaker/paper_143.pdf}\newline
\url{https://github.com/yemao616/Bayesian-Knowledge-Tracing}\newline
\url{http://www.cs.cmu.edu/~./ggordon/yudelson-koedinger-gordon-individualized-bayesian-knowledge-tracing.pdf}\newline
\url{https://www.fi.muni.cz/~xpelanek/publications/umuai-overview.pdf}\newline
\url{https://medium.com/@joyboseroy/modelling-a-students-learning-34375b0131dd}\newline
\url{https://www.math.vu.nl/~sbhulai/publications/data_analytics2018c.pdf}\newline
\subsection{Pronalaženje parametara}
\url{https://www.fmrib.ox.ac.uk/datasets/techrep/tr00yz1/tr00yz1/node9.html}\newline
\url{https://github.com/wlmiller/BKTSimulatedAnnealing/blob/master/computeKTparams_SA.java}\newline
\url{https://www.upenn.edu/learninganalytics/ryanbaker/paper_143.pdf}\newline
\url{https://educationaldatamining.org/files/conferences/EDM2018/papers/EDM2018_paper_14.pdf}\newline
\url{http://yudelson.info/hmm-scalable/}\newline
\url{https://www.educationaldatamining.org/EDM2015/proceedings/short364-367.pdf}\newline
\url{https://concord.org/wp-content/uploads/2016/12/pdf/tracking-student-progress-in-a-game-like-learning-environment.pdf}\newline
\url{https://tinyheero.github.io/2016/01/03/gmm-em.html}\newline
\url{https://machinelearningmastery.com/expectation-maximization-em-algorithm/}\newline
\url{http://rstudio-pubs-static.s3.amazonaws.com/1001_3177e85f5e4840be840c84452780db52.html}\newline
\url{https://www.colorado.edu/amath/sites/default/files/attached-files/em_algorithm.pdf}

\chapter{Zaključak}
Zaključak.

\bibliography{literatura}
\bibliographystyle{fer}

\begin{sazetak}
Sažetak na hrvatskom jeziku.

\kljucnerijeci{Ključne riječi, odvojene zarezima.}
\end{sazetak}

% TODO: Navedite naslov na engleskom jeziku.
\engtitle{Title}
\begin{abstract}
Abstract.

\keywords{Keywords.}
\end{abstract}

\end{document}

% TODO: Ne radi bibtex,jel mi treba?