
\chapter{Self-attentive knowledge tracing}
	Pronađen je rad koji istražuje novi smjer knowledge tracinga. Predloženi model SAKT prvo identificira relevantne koncepte znanja iz prošlih interakcija te predviđa korisnikove performanse na tim konceptima. SAKT daje težinske vrijednosti prethodno odogovorenim pitanjima. U radu se tvrdi da je prema AUC bolji u za 4.43\% od state-of-art metoda uprosječeno po svim korištenim skupovima podataka. Također, glavna komponenta (self-attention) se može paralelizirati što daje znatnu prednost po brzini naspram modela temeljenih na RNN-ovima.