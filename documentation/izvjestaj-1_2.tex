\documentclass{report}
\usepackage{titlesec}
\usepackage{hyperref} 
\usepackage[croatian]{babel}
\usepackage[nodayofweek,level]{datetime}
\usepackage[numbers,square]{natbib}
%\usepackage{cite}
%\renewcommand\citeleft{}  % no opening or closing brackets
%\renewcommand\citeright{}
\usepackage{etoolbox}
%\patchcmd{\thebibliography}{\section*{\refname}}{}{}{}
%start new chapter on the same page
\makeatletter
\patchcmd{\chapter}{\if@openright\cleardoublepage\else\clearpage\fi}{}{}{}
\makeatother

\titleformat{\chapter}[display]
  {\normalfont\bfseries}{}{0pt}{\Large}
\newcommand{\mydate}{\formatdate{13}{7}{2020}}

\begin{document}
\title{Uvod}
\selectlanguage{croatian}
\date{\mydate}
\maketitle

\noindent Početkom srpnja predstavljena je tema prakse „Umjetna inteligencija u adaptivnom učenju“. Definirani su zadaci proučavanja grafova znanja. Provedena je motivacija za osmišljavanjem vlastitih ideja i koncepata primjene adaptivnog učenja za e-learning platforme. Naglašena je potreba za istraživanjem fraza kao što su Bayesian Knowledge Tracing (BKT), Item Response Theory (IRT), Deep Knowledge Tracing (DKT), Intelligent Tutoring Systems (ITS), Massive Open Online Courses (MOOC), itd.

\chapter{Pojmovi i definicije}

Početna istraživanja uključivala su analizu generalne ideje adaptivnog učenja, kao i detaljno bavljenje teorijom iza grafova znanja.
\newline
\newline
Grafovi znanja (engl. \textit{knowledge graphs}) temelj su mnogim informacijskim sustavima koji zahtijevaju pristup strukturiranom znanju. Predstavljaju skup povezanih entiteta u obliku usmjerenog ili neusmjerenog grafa. Čvorovi su entiteti, a bridovi odnosi među njima.
\newline
\newline
Adaptivno učenje je edukacijska metoda koja koristi računalne algoritme kako bi se učenje prilagodilo jedinstvenim potrebama svakog učenika. Računala, na temelju odgovora ispitanika, prilagođavaju brzinu i sadržaj obrazovnog materijala pojedinom učeniku.

\chapter{Cilj}
Početni cilj projekta definiran je kao predočenje usmjerenog grafa znanja koji pokazuje odnose preduvjeta između različitih koncepata, kako bi korisnik nepoznati koncept naučio optimalnim putem ovisno o već poznatim konceptima. Korisnik pristupa inicijalnom ispitu i ovisno o rezultatu ispita usmjeren je na novi sadržaj (koji može biti željeni ciljni koncept ili koncept koji je sljedeći u grafu na putu do ciljnog) ili se vraća na onaj sadržaj iz kojega mu nedostaju vještine potrebne za usvajanje željenog koncepta.
\newline
\newline
Napretkom projekta, cilj je definiran kao sustav preporuke zadataka kako bi se korisniku omogućilo skraćivanje vremena potrebnog da savlada neko gradivo. 

\chapter{Postojeće ideje i koncepti}
Među prvima pronađena je i proučena platforma adaptivnog učenja Knewton korištena za personalizaciju edukacijskog sadržaja \citep{ct1}. 
\newline
\newline
Izazovnim područjem praćenja znanja (engl. \textit{knowledge tracing}) moguće je strojno modelirati znanje korisnika pomoću njegove interakcije s računalom tijekom procesa učenja \citep{ct2}. Korisnicima je predložen sadržaj učenja ovisno o njihovim potrebama (sadržaj je okarakteriziran kao prelagan, pretežak, moguće ga je potpuno preskočiti ili ostaviti za kasnije). Predviđaju se buduće korisničke performanse prema prošloj aktivnosti. Mnoge metode pri tome uključuju korištenje Markovljevih modela s ograničenom funkcionalnošću. Također, neki modeli koriste logističku regresiju uz PFA (eng. \textit{performance factors analysis}). U novije vrijeme, korištene su povratne neuronske mreže (engl. \textit{recurrent neural networks, RNN}) pa cjelokupni model nosi naziv Deep Knowledge Tracing (DKT). Posebice je popularna kompleksnija varijanta RNN-a, LSTM (eng.\textit{long short-term memory}). Interakcije (pitanje-odgovor) potrebno je pretvarati u vektore fiksne duljine (ideja je inpute predstaviti one-hot encodingom parova (oznaka zadatka, točnost)). Mapiranje u izlazne vrijednosti postiže se računom slijeda “skrivenih” stanja (uzastopnim enkodiranjem bitnih informacija proteklih zapažanja kako bi odgovarale budućim predikcijama). Izlazna vrijednost je vektor vjerojatnosti točnog rješavanja svakog od zadataka u modelu. 
Sposobnost predviđanja korisničkih performansi ispitana je na simuliranom skupu podataka. Umjetno su generirani korisnici koji rješavaju određen broj zadataka iz fiksnog skupa koncepata. Svaki korisnik ima latentnu “vještinu” za svaki koncept, a svaki zadatak ima koncept i težinu. Vjerojatnosti da korisnik točno riješi zadatak određene težine ako ima određenu vještinu modelirane su pomoću IRT-a. Također, otkriven je javno dostupan benchmark skup podataka, 2009-10 Assistments Data.
\newline
\newline
Ponađen je i proučen i sustav KnowEdu, kojem je cilj konstruirati grafove znanja za edukacijske potrebe i identificirati odnose između različitih koncepata \citep{ct3}.
Na početku se pokušavaju ustanoviti svi koncepti koji se nalaze u dostupnim nastavnim materijalima i tečajevima korištenjem strojnog učenja. Zatim se koriste CRF (engl. \textit{conditional random field}) model, povratna neuronska mreža i varijanta LSTM mreže GRU (engl. \textit{gated recurrent units}) kako bi se dobio graf preduvjeta. Skup podataka prikupljen je iz više testova pri čemu je svaki ispitanik riješio svaki test. Jedan test sastoji se od više pitanja koja ispituju znanje istog koncepta i pomoću njega se izračuna odgovarajuća ocjena znanja tog koncepta za svakog ispitanika. Ocjene predstavljaju vještine ispitanika u tom području i one su ulaz modela. Svaki model kao izlaz daje, za svaku kombinaciju koncepata, vjerojatnost da je koncept A preduvjet za znanje koncepta B.
\newline
\newline
U radu \citep{ct4} implementirana je Bayes ekstenzija IRT modela i ispitana na istim skupovima podataka kao i DKT metoda \cite{ct2}. Ustanovljeno je da se takav IRT model ponaša bolje nego DKT, što je dijelom objašnjeno time što se prilikom ranijeg DKT testiranja nisu micali duplikati, odnosno redci koji istu interakciju korisnika povezuju s više vještina. Time je DKT model više puta obrađivao istu interakciju i davao pojačana predviđanja (eng. \textit{boost}).
\newline
\newline
U radu \citep{ct5} predstavljen je regularizacijski član funkcije gubitka originalnog DKT modela čime se povećava konzistentnost u predikciji. Naglašeni su problemi dotadašnjih DKT modela - ponekad loše rekonstruiraju ulazno opažanje (gubitak uzima u obzir predikciju iduće interakcije, ali ne i trenutne) i ne uzima se u obzir očekivani napredak korisničkog znanja protokom vremena. Isto tako, napomenuta je nerealističnost BKT-a koji originalno pretpostavlja da ne postoji zaboravljanje i da su komponente znanja (vještine, koncepti, zadaci) međusobno neovisne. Kasnije je u BKT uvedena ideja pogađanja i slučajnog netočnog odgovaranja.
\newline  
\newline
Rad \citep{ct6} predstavlja novu metodu Latent Skill Embedding za personaliziranu preporuku nastavnih cjelina. Iako se ovim načinom ne stvara graf znanja, prate se ovisnosti sadržaja i korisniku se preporučuje prigodni sljedeći sadržaj, što u suštini odgovara zadatku grafa znanja u našem projektu. Ovaj vjerojatnosni model postavlja učenike, nastavne cjeline i procjene znanja u zajednički semantički prostor koji se naziva latentni prostor vještina. Učenik je predstavljen kao skup latentnih vještina različitih razina i on ima putanju kroz latentni prostor vještina. Cjeline i procjene znanja su statične i postavljene su na određenim mjestima; svaka cjelina prikazana je kao vektor dobitaka/prednosti koji ima svoj skup nužnih preduvjeta, a svaki test znanja prikazan je kao skup zahtjeva određene vještine.

\chapter{Predložene ideje i koncepti}
\begin{itemize}
\item[-]nabava/osmišljavanje pitanja - eksperti smo mi sami za početni model, a kasnije potrebna veća količina pitanja iz šireg područja (po mogućnosti već odgovorenih i grupiranih po učenicima/studentima, npr. matura, studentski kvizovi)
\item[-]čvorovi grafa - vještine, koncepti, “ciljevi učenja”
\item[-]više pitanja za određeni koncept, različite kategorije težina pitanja, moguće više pitanja za istu kategoriju težine
\item[-]lakša pitanja preduvjeti za odgovoriti teža, a pitanja iste kategorije moguće odgovoriti kojim god redoslijedom
\item[-]binarno označavanje točnih/netočnih odgovora ili pak račun vjerojatnosti; postoci važnosti točnog odgovaranja na pitanje za razumijevanje pojedinih koncepata
\item[-]težinska suma odgovora (kojoj teža pitanja pridonose više, a lakša manje) kao izračun trenutne ocjene znanja pojedinog koncepta, koja predstavlja ulaz u mrežu (broj ulaza = broj koncepata)
\item[-]izgradnja grafa - pomoću uvjetnih vjerojatnosti 
\end{itemize}

\chapter{Izazovi}
\begin{itemize}
\item[-]binarna reprezentacija korisničkog razumijevanja može biti nerealistična
\item[-]moguća višeznačnost zbog očekivanja većine modela da jedan koncept predstavlja jedan zadatak
\item[-]način predstavljanja pitanja (jedinstveni ID, tagovi) 
\item[-]važnost praćenja točnosti i redoslijeda zadavanja zadataka
\item[-]izlaz mreže - kako najjednostavnije prikazati graf znanja (npr. stablo odluke, matrica susjedstva…)
\item[-]razlikovanje različitih koncepata bez znanja eksperta
\item[-]kako s malim skupom podataka postići dobru reprezentaciju grafa, izbjeći prenaučenost
\item[-]oblikovanje zaboravljanja, pogađanja, napretka znanja protokom vremena
\end{itemize}

\chapter{Literatura}
%\bibliographystyle{plainnat}
\renewcommand{\bibsection}{}
\begin{thebibliography}{9}

\bibitem[1]{ct1} Knewton, \url{https://www.knewton.com/blog/mastery/what-are-knewtons-knowledge-graphs/}
\bibitem[2]{ct2} Deep Knowledge Tracing, \url{ http://papers.nips.cc/paper/5654-deep-knowledge-tracing.pdf}
\bibitem[3]{ct3} KnowEdu, \url{ https://ieeexplore.ieee.org/stamp/stamp.jsp?tp=&arnumber=8362657}
\bibitem[4]{ct4} Back to the basics: Bayesian extensions of IRT outperform
neural networks for proficiency estimation, \url{https://arxiv.org/pdf/1604.02336.pdf}
\bibitem[5]{ct5} Addressing Two Problems in Deep Knowledge Tracing via
Prediction-Consistent Regularization, \url{https://arxiv.org/pdf/1806.02180}
\bibitem[6]{ct6} Latent Skill Embedding (Lentil), \url{https://arxiv.org/pdf/1602.07029.pdf}

\end{thebibliography}




\end{document}
